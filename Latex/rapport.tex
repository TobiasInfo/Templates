%
\documentclass[]{article}
% Inclusion des packets
\usepackage[a4paper,top=2cm,bottom=1.5cm,left=2cm,right=2cm,marginparwidth=1.5cm]{geometry}
\usepackage[utf8]{inputenc}
\usepackage{graphicx}
\usepackage{amssymb,amsmath}
\usepackage[french]{babel}
\usepackage[T1]{fontenc}
\usepackage{fancyhdr}
\usepackage{setspace}
\usepackage[font=small, labelfont=bf]{caption}
\usepackage[colorlinks=true, allcolors=blue]{hyperref}
\usepackage{textcomp}
\usepackage{fancyvrb}
\usepackage{incgraph}
\usepackage{algorithm}
\usepackage{pdfpages}

% Mise en page 
\newcommand{\HRule}[1]{\rule{\linewidth}{#1}}
\onehalfspacing

\pagestyle{fancy}
\fancyhf{}
\setlength\headheight{15pt}
 \setlength{\marginparwidth}{2cm}
\usepackage[
    left = \flqq{},% 
    right = \frqq{},% 
    leftsub = \flq{},% 
    rightsub = \frq{} %
    ]{dirtytalk}
    \setlength\headheight{15pt}
    
    % Ligne de Haut de page (Nom + UV)
    \fancyhead[L]{SAVARY Tobias UTC - GI02} 
    \fancyhead[R]{TODO : UV - Rapport TPX}

    % Bas de page (Nomero de page + UTC)
    \fancyfoot[R]{\thepage}
    \fancyfoot[L]{Université de Technologie de Compiègne}
     \setlength {\marginparwidth}{2cm}
     \renewcommand{\footrulewidth}{.5pt}

\begin{document}

\title{ 
  % TODO : Possible de mettre des images en haut de la page de garde 
  % \begin{center}
  %         \includegraphics[width=5cm]{img/logo_IA01.png} \hspace*{1cm}
  %         \includegraphics[width=9cm]{img/logo_UTC.png} \\ [2cm]
  % \end{center}
		\HRule{2pt} \\

    % Titre du Rapport
		\LARGE \textbf{Rapport TPX:\\Titre 1 \\ Suite titre} 
		\HRule{2pt} \\ [5.5cm]


		\normalsize  
    % Nom de l'auteur
        \author{
            Tobias SAVARY \\[0.5cm]
           \\[1cm]
        }
		}
    % Année universitaire
		\maketitle
        \begin{center}
            Année universitaire 2022/2023
        \end{center}
\pagebreak

% Table des matières 
\tableofcontents

\pagebreak


\section{Introduction}

TODO : Entrez une introduction

\pagebreak

\section{Partie 1}

TODO : Entrez une partie

\begin{equation}
endettement = \frac{mensualités * 100}{revenus mensuels} \\
\end{equation}

\pagebreak

% Exemple d'equation

% \begin{equation}
% \frac{montant * (1+ \frac{taux}{100})}{durée * 12} \\
% \end{equation}

% exemple d'image

% \begin{figure}[H]
% \centering
% \includegraphics{img/fpret_possible_conso.jpg}
% \caption{TODO : description de l'image}
% \end{figure}

\section{Conclusion}

TODO: Conclusion

\end{document}
